\documentclass[a4paper]{article}

% Imports
\usepackage[utf8]{inputenc}
\usepackage[ngerman]{babel}
\usepackage[T1]{fontenc}
\usepackage{fancyhdr}
\usepackage{epigraph}
\usepackage{subfiles}
\usepackage[acronym,toc]{glossaries}
\usepackage{amsmath}
\usepackage{amssymb}
\usepackage{graphicx}
\usepackage{siunitx}
\usepackage{tkz-graph}
\usepackage[style=verbose]{biblatex}
\usepackage{hyperref}


\bibliography{../Bibliography}
\usetikzlibrary{positioning}

% Title page
\title{Exposé: Open Source}

\author{
  Blechschmidt, Til\\
  \texttt{til@blechschmidt.de}\\
  Nordakademie, Elmshorn
  \and
  Peeters, Noah\\
  \texttt{noah.peeters@icloud.com}\\
  Nordakademie, Elmshorn
}

% Numbering
\setcounter{secnumdepth}{3}
\setcounter{tocdepth}{2}

% Quote styling
\setlength\epigraphwidth{.8\textwidth}
\setlength\epigraphrule{0pt}

\linespread{1.25}

\begin{document}
    % Title page
	\maketitle
	
	\section{Einführung}
	   Je ein Satz zu den folgenden Punkten:
	   \begin{enumerate}
	       \item Ziel
	       \item Erwartung
	       \item Methodik
	       \item Fragestellung
	   \end{enumerate}
	
	   In der Arbeit soll die Frage, in wie weit Endanwender Open Source Software wahrnehmen, untersucht werden. Dabei soll nicht nur die \emph{direkte} Verwendung von Open Source Software, wie es zum Beispiel bei LibreOffice der Fall ist, eine Rolle spielen, sondern auch die \emph{indirekter} Nutzung als Komponente kommerzieller Lösungen wie Chromium bzw. WebKit als Basis von Google Chrome.
	   
	   Es wird erwartet, dass viele Nutzer unbewusst bzw. indirekt mit Open Source Software in Kontakt stehen, sie diese also als Teil größerer, kommerzieller Closed Source Software nutzen.
	   
	   % TODO mindestens zwei Wochen oder exakt zwei Wochen? Sie mal fragen
	   Als Grundlage wird eine quantitative Umfrage über mindestens 2 Wochen dienen, damit die aktuelle Situation möglichst akkurat in die Argumentation einfließt.
	
	\section{Umfrage}
		Zweck der Umfrage ist es einen repräsentativen Überblick über das Nutzerverhalten von Open Source Software nach demographischen Gruppen aufgeschlüsselt zu erhalten. Dabei soll primär Aufschluss über die Verteilung von indirekter zu direkter Nutzung von Open Source Software sowie der Kenntnisstand der Anwender über die Nutzung dieser gegeben werden. In diesem Kontext ist interessant, ob solche Anwendungen primär genutzt werden, weil sie Open Source sind oder diese aufgrund ihrer Qualität und ihrem Bekanntheitsgrad in gebrauch sind.
		Um ein möglichst unvoreingenommenes Ergebnis zu erhalten wird die Umfrage in drei Bereiche aufgeteilt:
	   
	   \paragraph{Bewusste Nutzung}
	       Die Umfrage wird mit abstrakten Fragen zur Nutzung von Open Source Software beginnen. Zunächst soll ermittelt werden, ob der Nutzer/die Nutzerin weiß, was Open Source Software ist.
	       
	       Um im weiteren Verlauf der Umfrage sinnvolle Antworten zu erhalten, wird an dieser Stelle die Definition von Software und von Open Source Software erklärt.
	       %TODO: Gegebenenfalls werden hier noch weitere Definitionen nach einem Test hinzugefügt.
	 
	   
	   \paragraph{Unbewusste Nutzung}
	       Im zweiten Abschnitt werden konkrete Beispiele für Open Source Software genannt, um den Nutzer/die Nutzerin  darauf Aufmerksam zu machen, wo überall Open Source Software vorkommt, ohne dass es ihm/ihr aufgefallen ist.
	       
	       Dabei wird sowohl nach direkter und indirekter Nutzung gefragt.
	       
	   \paragraph{Chancen und Risiken von Open Source Software}
	       Im dritten Abschnitt werden Fragen zum Thema Chancen und Risiken gestellt. Dazu wir gefragt, warum oder warum nicht Endnutzer oder Unternehmen Open Source Software einsetzten. Die möglichen Antworten werden aus der Schweizer Studie zum Thema Open Source\footcite{oss:studie} genommen, um die Wahrnehmung mit der realen Nutzung zu vergleichen.
	   \\\par
	   Um zu analysieren, ob es innerhalb der Bevölkerungen Differenzen gibt, soll die Umfrage eine Frage zur aktuellen Tätigkeit (Schüler, Student, Arbeitnehmer, etc.) und eine Frage zur Selbsteinschätzung der Computerkenntnisse enthalten.
	   
	   
	\section{Struktur}
	
    
    \clearpage
    \printbibliography

\end{document}
