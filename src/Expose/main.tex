\documentclass[a4paper]{article}

% Imports
\usepackage[utf8]{inputenc}
\usepackage[ngerman]{babel}
\usepackage[T1]{fontenc}
\usepackage{fancyhdr}
\usepackage{epigraph}
\usepackage{subfiles}
\usepackage[acronym,toc]{glossaries}
\usepackage{amsmath}
\usepackage{amssymb}
\usepackage{graphicx}
\usepackage{siunitx}
\usepackage{tkz-graph}
\usepackage[style=apa]{biblatex}
\usepackage{hyperref}


\bibliography{../Bibliography}
\usetikzlibrary{positioning}

% Title page
\title{Exposé Open Source}

\author{
  Blechschmidt, Til\\
  \texttt{til@blechschmidt.de}\\
  Nordakademie, Elmshorn
  \and
  Peeters, Noah\\
  \texttt{noah.peeters@icloud.com}\\
  Nordakademie, Elmshorn
}

% Numbering
\setcounter{secnumdepth}{3}
\setcounter{tocdepth}{2}

% Quote styling
\setlength\epigraphwidth{.8\textwidth}
\setlength\epigraphrule{0pt}

\linespread{1.25}

\begin{document}
    % Title page
	\maketitle
	
	\section{Einführung}
%		Je ein Satz zu den folgenden Punkten:
%		\begin{enumerate}
%		   \item Ziel
%		   \item Erwartung
%		   \item Methodik
%		   \item Fragestellung
%		\end{enumerate}
		
		In der Arbeit soll die Frage, in wie weit Endanwender Open Source Software wahrnehmen, untersucht werden. Dabei soll nicht nur die \emph{direkte} Verwendung von Open Source Software, wie es zum Beispiel bei LibreOffice der Fall ist, eine Rolle spielen, sondern auch die \emph{indirekter} Nutzung als Komponente kommerzieller Lösungen wie Chromium bzw. WebKit als Basis von Google Chrome.
		
		Es wird erwartet, dass viele Nutzer unbewusst bzw. indirekt mit Open Source Software in Kontakt stehen, sie diese also als Teil größerer, kommerzieller Closed Source Software nutzen.
		
		% TODO mindestens zwei Wochen oder exakt zwei Wochen? Sie mal fragen
		% TODO Why Umfrage?
		Als Grundlage wird eine quantitative Umfrage über mindestens 2 Wochen dienen, damit die aktuelle Situation möglichst akkurat in die Argumentation einfließt.
	
	\section{Umfrage}
		Zweck der Umfrage ist es einen repräsentativen Überblick über das Nutzerverhalten von Open Source Software nach demographischen Gruppen aufgeschlüsselt zu erhalten. Dabei soll primär Aufschluss über die Verteilung von indirekter zu direkter Nutzung von Open Source Software sowie der Kenntnisstand der Anwender über die Nutzung dieser gegeben werden. In diesem Kontext ist interessant, ob solche Anwendungen primär genutzt werden, weil sie Open Source sind oder diese aufgrund ihrer Qualität und ihrem Bekanntheitsgrad in Gebrauch sind. Ein weiterer Aspekt ist der Vergleich von der Sicht des Nutzers und der von Unternehmen auf Basis einer Schweizer Studie aus der Perspektive von Unternehmen\footcite{oss:studie}.\\
		% TODO Quantitative Umfrage/Warum geschlossene Fragen
		\subsection{Inhalt der Umfrage}
			Um ein möglichst unvoreingenommenes Ergebnis zu erhalten wird die Umfrage in vier Bereiche aufgeteilt:
		   
			\paragraph{Bewusste Nutzung}
				Die Umfrage wird mit abstrakten Fragen zur Nutzung von Open Source Software beginnen. Zunächst soll ermittelt werden, ob der Nutzer/die Nutzerin weiß, was Open Source Software ist.
			
				Um im weiteren Verlauf der Umfrage sinnvolle Antworten zu erhalten, wird an dieser Stelle die Definition von Software und von Open Source Software erklärt.
				% TODO: Gegebenenfalls werden hier noch weitere Definitionen nach einem Test hinzugefügt.
			
			
			\paragraph{Unbewusste Nutzung}
				Im zweiten Abschnitt werden konkrete Beispiele für Open Source Software genannt, um den Nutzer/die Nutzerin  darauf Aufmerksam zu machen, wo überall Open Source Software vorkommt, ohne dass es ihm/ihr aufgefallen ist. Dabei wird sowohl auf direkte als auch indirekter Nutzung eingegangen.
			
				Außerdem soll geklärt werden können, warum es Differenzen, wenn vorhanden, zwischen bewusster und unbewusster Nutzung gibt.
			
			\paragraph{Gründe für und gegen Open Source Software}
				Im dritten Abschnitt werden Fragen bezüglich der Nutzungs- und Hinderungsgründe von Open Source Software gestellt. Dazu wird gefragt, warum oder warum nicht Endnutzer oder Unternehmen Open Source Software einsetzten. Die möglichen Antworten werden aus der Schweizer Studie zum Thema Open Source\footcite{oss:studie} genommen, um die Wahrnehmung mit der realen Nutzung zu vergleichen.
			
			\paragraph{Demographische Differenzen}
				Um zu analysieren, ob es innerhalb der Bevölkerungen Differenzen gibt, soll die Umfrage eine Frage zur aktuellen Tätigkeit\footnote{Schüler, Student, Arbeitnehmer, etc.} und eine Frage zur Selbsteinschätzung der Computerkenntnisse enthalten.
				
		\subsection{Verbreitung}
			Um die Reichweite der Umfrage zu maximieren wird diese in Form einer Online-Befragung umgesetzt. Dabei werden Plattformen wie Social Media, E-Mail, Messenger genutzt. Zudem wird der Link zu der Umfrage über private Kontakte in Schulen und an der Nordakademie verbreitet. Zudem wird dadurch der Kostenaufwand minimiert.
	   
	   
	\section{Struktur}
    		Die Struktur der Arbeit ist wie folgt vorgesehen:
    
    		\paragraph{Was ist Open Source Software?}
    		Zunächst wird ein kurzer Überblick über Open Source Software gegeben. Dazu gehört die Definition und die heutige Verwendung.
		
		\paragraph{Nutzung von Open Source Software von Endanwendern}
		Im nächsten Abschnitt wird die Nutzung von Open Source Software durch Endanwendern analysiert. Dabei wird zum einen betrachtet, ob und wie Endanwender wahrnehmen, dass sie Open Source Software direkt und indirekt nutzen.
		Zum anderen wird betrachtet, warum Endanwender Open Source Software nutzen.
		
		\paragraph{Gründe für und gegen die Nutzung von Open Source Software aus der Sicht von Endanwendern}
		Im letzten Abschnitt wird untersucht, wie Endanwender die Vor- und Nachteile von Open Source Software Nutzung einschätzen.
		Diese Einschätzung wird für Unternehmen mit der Realität verglichen.
		Außerdem wird untersucht, ob es bei der Einschätzung bei Unternehmen und bei Endanwendern Unterschiede gibt.
	
    \clearpage
    \nocite{*}
    \printbibliography

\end{document}
