\documentclass[a4paper]{article}

% Imports
\usepackage[utf8]{inputenc}
\usepackage[ngerman]{babel}
\usepackage[T1]{fontenc}
\usepackage{fancyhdr}
\usepackage{epigraph}
\usepackage{subfiles}
\usepackage[acronym,toc]{glossaries}
\usepackage{amsmath}
\usepackage{amssymb}
\usepackage{graphicx}
\usepackage{siunitx}
\usepackage{tkz-graph}
\usepackage[style=apa]{biblatex}
\usepackage{hyperref}


\bibliography{./Bibliography}
\usetikzlibrary{positioning}

% Title page
\title{Exposé Open Source}

\author{
  Blechschmidt, Til\\
  \texttt{til@blechschmidt.de}\\
  Nordakademie, Elmshorn
  \and
  Peeters, Noah\\
  \texttt{noah.peeters@icloud.com}\\
  Nordakademie, Elmshorn
}

% Numbering
\setcounter{secnumdepth}{3}
\setcounter{tocdepth}{2}

% Quote styling
\setlength\epigraphwidth{.8\textwidth}
\setlength\epigraphrule{0pt}

\linespread{1.25}

\begin{document}
    % Title page
	\maketitle
	
	\section{Einführung}
		In der Arbeit soll die Frage, inwiefern Endanwender Open Source Software wahrnehmen, untersucht werden. Dabei soll nicht nur die \emph{direkte} Verwendung von Open Source Software, wie es zum Beispiel bei LibreOffice der Fall ist, eine Rolle spielen, sondern auch die \emph{indirekte} Nutzung als Komponente kommerzieller Lösungen wie Chromium bzw. WebKit als Basis von Google Chrome\footcite{is:open:source:right:for:you} aber auch als Werkzeug wie Linux als Betriebssystem für Server\footcite{report:BaselineScenario}.
		
		Es wird erwartet, dass viele Nutzer unbewusst und indirekt mit Open Source Software in Kontakt stehen, sie diese also als Teil größerer, kommerzieller Closed Source Software nutzen, sich dessen aber nicht bewusst sind. Außerdem ist zu erwarten, dass nur ein kleiner Teil der Anwender direkt mit Open Source Software interagiert\footcite{oss:lotus-to-linux}.
		
		Um zuverlässige Daten zu erhalten, die die aktuelle Wahrnehmung der Endanwender widerspiegeln, wird eine Umfrage durchgeführt. Für die Sicherstellung der Repräsentativität der Umfrage werden mindestens 80 Antworten erwartet.
		Die Umfrage wird quantitativ mit geschlossenen Fragen gestaltet, um den Aufwand für die Befragten zu minimieren und damit die Teilnehmerrate und Datenmenge zu maximieren.\footcite{handbuch:methoden}
	
	\section{Umfrage}
		Zweck der Befragung ist es, einen repräsentativen Überblick über die Wahrnehmung von Open Source Software des Endanwenders zu erhalten. Dabei soll Aufschluss über die Verteilung von indirekter sowie direkter Nutzung gegeben werden und in dem Zusammenhang analysiert werden, warum sich Anwender für Open Source Software entscheiden. Außerdem soll evaluiert werden, welche Gründe die Nutzer für und gegen die Nutzung von Open Source Software sehen.
	
		\subsection{Inhalt der Umfrage}
			Die Erhebung ist in vier Teile gegliedert, die im Folgenden näher erläutert sind: 
		   
			\paragraph{Bewusste Nutzung}
				Um die, dem Anwender bewusste, Nutzung von Open Source Software zu erheben, beginnt die Umfrage mit diesem Abschnitt.\\
				Zunächst wird ermittelt, ob die Nutzer wissen, was Open Source Software ist. Um im weiteren Verlauf der Umfrage sinnvolle Antworten zu erhalten, wird anschließend die Definition von Software sowie quelloffener Software erklärt.\\
				Nun werden Fragen zum Nutzungsverhalten von solcher Software gestellt ohne dabei Beispiele zu nennen, um Beeinflussung zu vermeiden.
			
			\paragraph{Unbewusste Nutzung}
				Im zweiten Abschnitt werden konkrete Beispiele für Open Source Software genannt, um die Nutzer darauf aufmerksam zu machen, wo Open Source Komponenten und Softwares überall eingesetzt werden, ohne dass es ihnen klar ist.\\
				Außerdem soll geklärt werden, warum es Differenzen, wenn vorhanden, zwischen bewusster und unbewusster Nutzung gibt.
			
			\paragraph{Gründe für und gegen Open Source Software}
				Im dritten Abschnitt werden Fragen bezüglich der Nutzungs- und Hinderungsgründe von Open Source Software gestellt. Dazu wird gefragt, warum oder warum nicht Endnutzer oder Unternehmen Open Source Software einsetzten. Die möglichen Antworten werden aus der Schweizer Studie zum Thema Open Source\footcite{oss:studie} genommen, um die Wahrnehmung mit der realen Nutzung vergleichbar zu machen.
			
			\paragraph{Demographische Differenzen}
				Zur Analyse von Differenzen innerhalb der Bevölkerungen enthält die Umfrage Fragen zur aktuellen Tätigkeit\footnote{Schüler, Student, Arbeitnehmer, etc.}, dem Geschlecht sowie zur Selbsteinschätzung der Computerkenntnisse.
				
		\subsection{Verbreitung}
			Um die Reichweite der Umfrage zu maximieren und die Kosten zu minimieren wird diese in Form einer Onlinebefragung umgesetzt. Dabei werden Kanäle wie Social Media, E-Mail, Messenger genutzt, um auf die Umfrage aufmerksam zu machen. Zudem wird der Link zu der Umfrage über private Kontakte in Schulen und an der Nordakademie verbreitet.
	   
	   
	\section{Struktur}
		Die Struktur der Arbeit ist wie folgt vorgesehen:

		\paragraph{Was ist Open Source Software?}
			Zunächst wird ein kurzer Überblick über Open Source Software gegeben. Dazu gehören die Definition, als Software deren Quellcode öffentlich zugänglich ist und weiteren Kriterien \footcite{BAHAMDAIN2015459}, und die heutige Verwendung.
		
		\paragraph{Nutzung von Open Source Software von Endanwendern}
			Im nächsten Abschnitt wird die Nutzung von Open Source Software durch Endanwender analysiert. Dabei wird zum einen betrachtet, ob und wie Endanwender wahrnehmen, dass sie Open Source Software direkt und indirekt nutzen.
			Zum anderen wird betrachtet, warum Endanwender sich bewusst für Open Source Software entscheiden.
		
		\paragraph{Gründe für und gegen die Nutzung von Open Source Software aus Sicht der Endanwender}
			Im letzten Abschnitt wird untersucht, wie Endanwender die Vor- und Nachteile von Open Source Software Nutzung einschätzen.
			Diese Einschätzung wird für Unternehmen mit der Realität verglichen.
			Außerdem wird untersucht, ob es bei der Einschätzung bei Unternehmen und bei Endanwendern Unterschiede gibt.
	
    \clearpage
    \nocite{*}
    \printbibliography

\end{document}
