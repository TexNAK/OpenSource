\documentclass[a4paper]{article}
% \documentclass[a4paper]{book}

% Imports
\usepackage[utf8]{inputenc}
\usepackage[ngerman]{babel}
\usepackage[T1]{fontenc}
\usepackage{fancyhdr}
\usepackage{epigraph}
\usepackage{subfiles}
\usepackage[acronym,toc]{glossaries}
\usepackage{graphicx}
\usepackage[style=apa]{biblatex}
\usepackage{hyperref}

\bibliography{../Bibliography}

% Generate the glossary
\makeglossaries

% \newglossaryentry{oss}{name=OSS, description={Open Source Software}}
\newacronym{oss}{OSS}{Open Source Software}

% Title page
\title{Open Source Software \\
    \noindent\rule[0.25ex]{\linewidth}{0.5pt}
    \large Wie nehmen Endanwender Open Source Software wahr?
}

\author{
  Blechschmidt, Til\\
  \texttt{til@blechschmidt.de}\\
  Nordakademie, Elmshorn
  \and
  Peeters, Noah\\
  \texttt{noah.peeters@icloud.com}\\
  Nordakademie, Elmshorn
}

% Numbering
\setcounter{secnumdepth}{3}
\setcounter{tocdepth}{2}

% Quote styling
\setlength\epigraphwidth{.8\textwidth}
\setlength\epigraphrule{0pt}


\begin{document}
    % Title page
    \thispagestyle{fancy}
    \maketitle

    % Abstract
    \begin{abstract}
         OSS Abstract
    \end{abstract}
    \newpage

    % TOC
    \tableofcontents
    \clearpage

    % --- Content ---
    \section{Einführung}
        \subsection{Open Source Software}
            \subsubsection{Definition}
                \begin{quote} 
                    \centering 
                    Software wird als Open Source bezeichnet, wenn ihr Quelltext frei zugänglich ist. Sie kann in kommerzieller Software eingesetzt werden, ihre Nutzung muss allerdings nicht kostenfrei sein. 
                \end{quote}
        
    \section{Umfrageauswertung}
        \subsection{Test}
            \gls{oss} und \gls{oss}
    
    \section{Zusammenfassung}
    
    \clearpage
    \section*{Appendix}
    
        \printglossary[type=\acronymtype]
        \printglossary
        
        \clearpage
        \nocite{*}
        \printbibliography
        
        \listoffigures

\end{document}
